
\section{Notas de Performance}

A peça tem dois acordes estruturantes. O primeiro é extraído de uma análise espectral dos sinos da Comunidade do Redentor (Curitiba, PR). O segundo, é uma modulação por anel das frequências do primeiro. O primeiro acorde é preponderante no início da peça (seção A); nas seções B e C o segundo acorde é incluído aos poucos; a seção D usa apenas o segundo acorde; e a seçao final E apresenta a sobreposição dos dois acordes.

A parte eletrônica apresenta as frequências destes acordes assim como foram analisadas (incluindo seus microintervalos), enquanto que a para o órgão as frequências foram arredondadas para se conformarem ao temperamento igual. A  pequena diferença entre as alturas de um meio e de outro é  um elemento importante para a peça. Ela deve causar um efeito de batimentos pela interferência de uma frequência sobre a outra. 

A peça apresenta intensas mudanças de andamento que têm por objetivo proporcionar dois tipos de escuta: uma que se concentra nos aspectos internos dos sons (movimentos, batimentos, microintervalos, textura sonora), privilegiada por um desenvolvimento mais lento no tempo (\textit{Quasi Statico}); e outra direcionada aos gestos, saltos melódicos, motivos, que é favorecida por um maior movimento no tempo (\textit{Lento, Andante}).


Registros

Esta peça foi composta tendo em mente o órgão no qual será estreada (Comunidade do Redentor). A notação de troca de registros seguiu a prática do grupo \textit{Ars Iubilorum} para esse instrumento. O órgão possui apenas manual com registros para notas acima do Dó central e registros para notas abaixo do Dó central

\vspace{10mm}
\scalebox{1.2}{
\setlength{\unitlength}{0.8cm}
\begin{picture}(12,4)
\thicklines
\put(1,4){{\footnotesize lado esquerdo}}
\put(4.2,4){{\footnotesize lado direito}}

\put(3.6,3.7){\circle*{0.2}}
\put(3.9,3.7){\circle*{0.2}}
\put(3.1,3.6){{\footnotesize 8'}}
\put(4.2,3.6){{\footnotesize 8'}}

\put(3.6,3.4){\circle{0.2}}
\put(3.9,3.4){\circle{0.2}}
\put(3.1,3.3){{\footnotesize 4'}}
\put(4.2,3.3){{\footnotesize 4'}}

\put(3.6,3.1){\circle{0.2}}
\put(3.9,3.1){\circle{0.2}}
\put(3.1,3){{\footnotesize 2'}}
\put(4.2,3){{\footnotesize 2'}}

\put(3.6,2.8){\circle{0.2}}
\put(3.9,2.8){\circle{0.2}}
\put(2.4,2.7){{\footnotesize 1 1/3'}}
\put(4.2,2.7){{\footnotesize 1 1/3'}}

\put(1,2.2){\circle*{0.2}}
\put(1.3 , 2.1){{\footnotesize : registro ativo}}
\put(1,1.8){\circle{0.2}}
\put(1.3 , 1.7){{\footnotesize : registro inativo}}
\end{picture}
}

Caso seja tocada em outro órgão, podemos explorar outras possibilidades de registração em diálogo.



