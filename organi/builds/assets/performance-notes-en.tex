
\section{Performance Notes}

The piece has two structuring chords. The first is extracted from a spectral analysis of the bells of the Redeemer Community (Curitiba, PR). The second is a ring modulation of the first. At the beginning of the piece (section A), the first chord is preponderant. In sections B and C, the second chord is gradually added. Section D uses only the second chord. Finally, the E section presents the superposition of the two chords.

The electronic part presents the frequencies of these chords as analyzed (including their microintervals), while for the organ the frequencies have been rounded to conform to the equal temperament. The small difference between the frequencies in one medium and in another is an important element of the piece. It must cause a beating effect.

The piece features intense tempo changes that aim to provide two types of listening: one that focuses on the internal aspects of sounds (movements, beats, micro-intervals, sound texture), favored by a slower development in time (\textit{Quasi static}); and another directed to gestures, melodic jumps, motifs, which is favored by a greater movement in time (\textit{Lento, Andante}).


Records

This piece was composed keeping in mind the organ in which it will be premiered (Redeemer Community). The registration followed the practice of the \textit{Ars Iubilorum} group for this instrument. The organ only has a manual with records for notes above central C and records for notes below central C.

\vspace{10mm}
\scalebox{1.2}{
\setlength{\unitlength}{0.8cm}
\begin{picture}(12,4)
\thicklines
\put(1,4){{\footnotesize left side}}
\put(4.2,4){{\footnotesize right side}}

\put(3.6,3.7){\circle*{0.2}}
\put(3.9,3.7){\circle*{0.2}}
\put(3.1,3.6){{\footnotesize 8'}}
\put(4.2,3.6){{\footnotesize 8'}}

\put(3.6,3.4){\circle{0.2}}
\put(3.9,3.4){\circle{0.2}}
\put(3.1,3.3){{\footnotesize 4'}}
\put(4.2,3.3){{\footnotesize 4'}}

\put(3.6,3.1){\circle{0.2}}
\put(3.9,3.1){\circle{0.2}}
\put(3.1,3){{\footnotesize 2'}}
\put(4.2,3){{\footnotesize 2'}}

\put(3.6,2.8){\circle{0.2}}
\put(3.9,2.8){\circle{0.2}}
\put(2.4,2.7){{\footnotesize 1 1/3'}}
\put(4.2,2.7){{\footnotesize 1 1/3'}}

\put(1,2.2){\circle*{0.2}}
\put(1.3 , 2.1){{\footnotesize : active register}}
\put(1,1.8){\circle{0.2}}
\put(1.3 , 1.7){{\footnotesize : inactive register}}
\end{picture}
}

If played in another organ, we can explore other registration options.


